\documentclass[10pt,a4paper]{article}

\input{classicthesis-proposal-config}


% Set these values
\newcommand{\myTitle}{Title: The one-line description of my Master thesis\xspace}

\newcommand{\myName}{Name\xspace}
\newcommand{\myUni}{University\xspace}
\newcommand{\myLocation}{City, Country\xspace}
\newcommand{\myEmail}{mail@univeristy.com\xspace}

\newcommand{\myAdvisorName}{Advisor Name\xspace}
\newcommand{\myAdvisorUni}{Advisor University\xspace}
\newcommand{\myAdvisorLocation}{Advisor City, Country\xspace}
\newcommand{\myAdvisorEmail}{advisor@univeristy.com\xspace}


% DO NOT CHANGE  THESE
%==============
\hypersetup{%
    colorlinks=true, linktocpage=true, pdfstartpage=3, pdfstartview=FitV,%
    breaklinks=true, pageanchor=true,%
    pdfpagemode=UseNone, %
    plainpages=false, bookmarksnumbered, bookmarksopen=true, bookmarksopenlevel=1,%
    hypertexnames=true, pdfhighlight=/O,%
    urlcolor=CTurl, linkcolor=CTlink, citecolor=CTcitation, %
    pdftitle={\myTitle},%
    pdfauthor={\textcopyright\ \myName},%
    pdfsubject={},%
    pdfkeywords={},%
    pdfcreator={pdfLaTeX},%
    pdfproducer={LaTeX with hyperref and classicthesis}%
}

\addbibresource{References.bib}

\begin{document}
    \pagestyle{plain}
    \title{\color{CTtitle}\rmfamily\normalfont\spacedallcaps{\myTitle}}
    \author{ \normalsize \spacedallcaps{\myName} }
    \affil{ \vspace{-8pt} \small \myUni, \\ \myLocation \\ {\em \myEmail} }
    \author{ \medskip  {\small \em Advisor} \\ \smallskip {\normalsize \myAdvisorName}}
    \affil{ \vspace{-8pt} \small \myAdvisorUni, \\ \myAdvisorLocation \\ {\em \myAdvisorEmail}}
    \date{ \medskip \small \today} 
    \maketitle    
    \medskip
    
    % START EDITING FROM BELOW
    %==============
        
    \begin{abstract}
        \noindent 
        %Write abstract below
        \lipsum[1] Just a test. \footnote{This is a footnote.}
    \end{abstract}
    
    \section{Introduction}
    % Problem domain (State of the world...)
    % Issue or Research gap (the big BUT...)
    % Proposal (therefore I did...)
    % Key findings (I found...)
    % Contributions (what I am giving to research community...)
    Test citation \cite{feynman:1985}
    \lipsum[1-2]
    
    \section{Literature Review}
    % Relevant research in last 5 year (only list seminal works from more than 5 years ago)
    % Group relevant works into themes (subsections)
    % In paragraphs, each para stating
        % What did they do?
        % What did they find?
        % How is yours different?
    \lipsum[2]
    
    \subsection{Theme 1}
    \lipsum[7-8]
        
    \subsection{Theme 2}
    \lipsum[1-3]
    
    \subsection{Theme 3}
    \lipsum[3-5]
    
    \section{Research Gap}
    % In few lines, summarize the research gap from literature review
    \lipsum[4][1-6]
    
    \section{Aim of the Thesis}
    % In few lines, specify the aim of the thesis and how you are scoping it
    \lipsum[5][2-6]
    
    \section{Research Questions \& Contributions}
    % In a list, point out your research questions and contributions, usually 2 to 3
    \lipsum[1][1-4]
    \begin{description}
        \item[\textit{RQ1}] \lipsum[2][1]
        \item[\textit{RQ2}] \lipsum[3][1]
    \end{description}

    \lipsum[2][1-4]
    \begin{description}
        \item[\textit{RC1}] \lipsum[2][1]
        \item[\textit{RC2}] \lipsum[3][1]
    \end{description}
       
    \section{Proposed Solution}
    %  List or describe your proposed solution. Figures will help.
    \lipsum[1-4]
    
    \begin{figure}[!hbt]
        \centering
        \includegraphics[width=0.7\linewidth]{gfx/example_1}
        \caption{This is an example figure}
        \label{fig:example1}
    \end{figure}    
    
    \section{Evaluation}
    % How do you justify that your system works.
    % Usually, in two ways: system evaluation (accuracy, precision..) and human evaluation (usability)
    \lipsum[1-1]
    
    \section{Benefit}
    %-who is it benefiting and how ?
    \lipsum[1-1]
    
    
    
    \section{Timeline}
    % Your proposed timeline of how you will finish your thesis within 6 months
    
    \begin{table}[!hbt]
        \begin{tabularx}{\textwidth}{p{50mm} *{12}{W}}		
            \noalign{\hrule height 1pt}
            \centering \multirow{2}{*}{Activity}&\multicolumn{12}{c}{Months}\\
            \cline{2-13}
            &1&&2&&3&&4&&5&&6\\
            \hline
            1. Literature Review & \cellcolor{CTtitle} &   &   &   &   &   &   &   &   &   &   &   \\ 
            2. Preliminary study&&\multicolumn{2}{c}{\cellcolor{CTtitle}}&&&\\
            3. Dataset generation&&&&\multicolumn{3}{c}{\cellcolor{CTtitle}}&\\
            4. Development&&&&&&&\multicolumn{3}{c}{\cellcolor{CTtitle}}&\\
            5. Evaluation&&&&&&\multicolumn{5}{c}{\cellcolor{CTtitle}}\\
            6. Documentation&&&&&&&&\multicolumn{4}{c}{\cellcolor{CTtitle}}\\
            \noalign{\hrule height 1pt}
        \end{tabularx}
    \end{table}
    
    
    \section{Conclusion}
    % Affirm how you are going to deliver your claims
    % Summarize your contribution
    % Key points to take away
    % Report as if it is a press release of your work
    \lipsum[1-1]
    
    %    DO NOT CHANGE BELOW
    
    \defbibheading{bibintoc}[\bibname]{%
        \manualmark
        \markboth{\spacedlowsmallcaps{References}}{\spacedlowsmallcaps{References}}%
        \section*{References}%
    }
    \printbibliography[heading=bibintoc]
    
    
    
\end{document}